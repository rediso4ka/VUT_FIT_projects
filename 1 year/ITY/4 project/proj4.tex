\documentclass[12pt, a4paper]{article}

\usepackage[utf8]{inputenc}
\usepackage[left=2cm,text={17cm,24cm},top=3cm]{geometry}
\usepackage[IL2]{fontenc}
\usepackage{times}
\usepackage[czech]{babel}
\usepackage{url}
\usepackage[unicode]{hyperref}
\usepackage[hyphenbreaks]{breakurl}


\begin{document}

\begin{titlepage}
\begin{center}
    \huge
    \textsc{\Huge Vysoké učení technické v Brně\\
    \LARGE Fakulta informačních technologií}\\
    \vspace{\stretch{0.382}}
    \LARGE Typografie a~publikování\,--\,4. projekt\\
    \Huge Japonský knihtisk\\
    \vspace{\stretch{0.618}}
\end{center}
{\Large \today \hfill
Aleksandr Shevchenko}
\end{titlepage}

\section{Vznik knihtisku v Japonsku}

Japonský knihtisk má dlouhou historii. Japonsko bylo jednou ze zemí, které jako první přebraly tuto dovednost u Číňanů. Jak uvádí \cite{Egorova2015}, knihtisk se objevil v této zemi na začátku 8. století během vládnutí císařky Koken.

\section{Směr tisku}
Japonsko nemělo svůj systém psání, než ho přebralo u Činska v 1. století. Pak už v roce 650 Japonci používali písmena, importovaná z Číny, píše \cite{Skritter2014}. 

Japonci umí číst a psát nejen horizontálně, ale vertikálně. Historicky čtou sloupce textů shora dolů, zprava doleva, ale od roku 1949 všechny oficiální doklady musí být v horizontálním formátu. Podrobnější informace jsou v \cite{Osaka1991}.

\section{Manga}
\subsection{Vznik}
Podle \cite{Mccarthy2014}, dějiny \emph{mangy} (grafický komiks) začínají již s začátkem knihtisku v Japonsku\,--\,v 8. století. Samozřejmě, tehdy to byly jen jednoduché karikatury. Známý nám formát mangy vznikl po Druhé světové válce. S odchodem militaristické vlády přišly velké změny, což způsobilo vznik těchto kreativních knih\,--\,viz \cite{Jays2016}.

\subsection{Šíření}
Jak uvádí \cite{Gyllenfjell2013}, když v roce 1988 manga \uv{\emph{Akira}} se dostala do Severní Ameriky a Evropy, měla velký úspěch a hned se stala kultem. Popularita mangy se zvětšila díky \uv{\emph{Sailor Moon}} a \uv{\emph{Dragon Ball}}, které měly svou adaptaci i na TV.

\subsection{Uctívání tradicím}
Manga není výjimkou z tradičního pravidla japonského tisku, v ní se taky dodržuje vertikální formát čtení. Nevadí, jestli je přeložena do cizího jazyku\,--\,dodržení tohoto formátu dává možnost čtenářům seznámit se blíž s japonskou kulturou (viz \cite{Naveen2021}).

\subsection{Dnešní doba}
Jak uvádí \cite{Macwilliams2014}, čtení mangy je jednou z nejdůležitějších částí každodenní rutiny v Japonsku. Manga se především odlišuje počtem obrázků. Čtenář je schopen pochopit téma mangy i bez velkého množství slov\,--\,viz \cite{Ito2020}.

V současné době čtení zůstává jedním z nejpopulárnějších koníčků Japonců. Většina knih je dostupna ve formátu pocketbooků (\emph{bunkobon})\,--\,viz \cite{Andrew2011}.

\newpage
\renewcommand{\refname}{Literatura}
\bibliographystyle{czechiso}
\bibliography{bibliography}

\end{document}
