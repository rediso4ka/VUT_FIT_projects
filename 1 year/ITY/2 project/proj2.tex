\documentclass[11pt, a4paper, twocolumn]{article}

\usepackage[left=1.5cm,text={18cm, 25cm},top=2.5cm]{geometry}
\usepackage[utf8]{inputenc}
\usepackage[czech]{babel}
\usepackage[IL2]{fontenc}
\usepackage{times}
\usepackage{amsthm}
\usepackage{amssymb}
\usepackage{amsmath}

\begin{document}

\begin{titlepage}
\begin{center}
\huge
\textsc{\Huge Vysoké učení technické v Brně\\
\huge Fakulta informačních technologií}\\
\vspace{\stretch{0.382}}
\LARGE Typografie a publikování\,--\,2. projekt\\
Sazba dokumentů a matematických výrazů\\
\vspace{\stretch{0.618}}
\end{center}
{\Large 2022 \hfill
Aleksandr Shevchenko (xshevc01)}
\end{titlepage}
%\newpage
\section*{Úvod}

V~této úloze si vyzkoušíme sazbu titulní strany, matematických vzorců, prostředí a~dalších textových struktur obvyklých pro technicky zaměřené texty (například rovnice (\ref{dva}) nebo Definice \ref{def} na straně \pageref{def}). Pro vytvoření těchto odkazů používáme příkazy \verb|\label|, \verb|\ref| a~\verb|\pageref|.

Na titulní straně je využito sázení nadpisu podle optického středu s~využitím zlatého řezu. Tento postup byl probírán na přednášce. Dále je na titulní straně použito odřádkování se zadanou relativní velikostí 0,4 em a~0,3~em.


\section{Matematický text}

Nejprve se podíváme na sázení matematických symbolů a~výrazů v~plynulém textu včetně sazby definic a~vět s~využitím balíku \verb|amsthm|. Rovněž použijeme poznámku pod čarou s~použitím příkazu \verb|\footnote|. Někdy je vhodné použít konstrukci \verb|${}$| nebo \verb|\mbox{}|, která říká, že (matematický) text nemá být zalomen. 

\theoremstyle{definition}
\newtheorem{defn}{Definice}
\begin{defn}Nedeterministický Turingův stroj \emph{(NTS) je šestice tvaru  $M = (Q,\Sigma,\Gamma,\delta,q_{0},q_{F})$, kde:} 
\begin{itemize}
    \item $Q$ \emph{je konečná množina} vnitřních (řídicích) stavů\emph{,}
    \item $\Sigma$ \emph{je konečná množina symbolů nazývaná} vstupní abeceda\emph{, $\Delta \not\in \Sigma$,}
    \item $\Gamma$ \emph{je konečná množina symbolů,} $\Sigma \subset \Gamma$\emph{,} $\Delta \in \Gamma$ \emph{, nazývaná} pásková abeceda\emph{,}
    \item $\delta : (Q\ \backslash\ \{q_{F}\})\times \Gamma \rightarrow 2^{Q\times(\Gamma\cup\{L,R\})}$ \emph{, kde $L, R \not\in \Gamma$, je parciální} přechodová funkce\emph{, a}
    \item $q_{0} \in  Q$ \emph{je} počáteční stav \emph{a}~$q_{F} \in  Q$ \emph{je} koncový stav.
\end{itemize}
\end{defn}
Symbol $\Delta$ značí tzv. \emph{blank} (prázdný symbol), který se vyskytuje na místech pásky, která nebyla ještě použita.

\emph{Konfigurace pásky} se skládá z~nekonečného řetězce, který reprezentuje obsah pásky, a~pozice hlavy na tomto řetězci. Jedná se o~prvek množiny $\{\gamma\Delta^{\omega}\ |\ \gamma\in\Gamma^{*}\}\times\mathbb{N}$\footnote{Pro libovolnou abecedu $\Sigma$ je $\Sigma^{\omega}$ množina všech \emph{nekonečných} řetězců nad $\Sigma$, tj. nekonečných posloupností symbolů ze $\Sigma$.}. \emph{Konfiguraci pásky} obvykle zapisujeme jako ${\Delta xyz\underline{z}x\Delta\dots}$ (podtržení značí pozici hlavy). \emph{Konfigurace stroje} je pak dána stavem řízení a~konfigurací pásky. Formálně se jedná o~prvek množiny $Q \times \{\gamma\Delta^{\omega}\ | \  \gamma\in\Gamma^{*}\}\times\mathbb{N}$.

\subsection{Podsekce obsahující definici a větu}
\begin{defn}\label{def}Řetězec $w$ nad abecedou $\Sigma$ je přijat NTS~$M$\emph{,} \emph{jestliže $M$ při aktivaci z~počáteční konfigurace pásky $\underline{\Delta}w\Delta\dots$ a~počátečního stavu $q_{0}$ může zastavit přechodem do koncového stavu $q_{F}$, tj. $(q_{0},\Delta w\Delta^{\omega},0) \underset{M}{\overset{*}{\vdash}} (q_{F},\gamma,n)$ pro nějaké $\gamma\in\Gamma^{*}$ a~$n\in\mathbb{N}$.}

\emph{Množinu $L(M) = \{w\ |\ w$ je přijat NTS $M\} \subseteq \Sigma^{*}$ nazýváme} jazyk přijímaný NTS $M$.
\end{defn}

Nyní si vyzkoušíme sazbu vět a~důkazů opět s~použitím balíku \verb|amsthm|.
\newtheorem{lem}{Věta}
\begin{lem}\emph{Třída jazyků, které jsou přijímány NTS, odpovídá} rekurzivně vyčíslitelným jazykům.
\end{lem}

\section{Rovnice}

Složitější matematické formulace sázíme mimo plynulý text. Lze umístit několik výrazů na jeden řádek, ale pak je třeba tyto vhodně oddělit, například příkazem \verb|\quad|.

\[x^{2} - \sqrt[4]{y_{1}*y_{2}^{3}}\quad  x>y_{1}\geq y_{2}\quad z_{z_{z}}\neq \alpha_{1}^{\alpha_{2}^{\alpha_{3}}}\]

V~rovnici (\ref{jeden}) jsou využity tři typy závorek s~různou explicitně definovanou velikostí.

\begin{eqnarray}
\label{jeden}x &=& \bigg\{a\oplus \Big[b\cdot\big(c\ominus d\big)\Big]\bigg\}^{4/2}
\\
\label{dva}y &=& \lim_{\beta\to\infty}\frac{\tan^{2}\beta - \sin^{3}\beta}{\frac{1}{\frac{1}{\log_{42} x}+\frac{1}{2}}}
\end{eqnarray}

V~této větě vidíme, jak vypadá implicitní vysázení limity $\lim_{n\to\infty}f(n)$ v~normálním odstavci textu. Podobně je to i~s~dalšími symboly jako $\bigcup_{N\in\mathcal{M}}N$ či $\sum^{n}_{j=0}x^{2}_{j}$. 
S~vynucením méně úsporné sazby příkazem \verb|\limits| budou vzorce vysázeny v~podobě $\lim\limits_{n\to\infty}f(n)$ a~$\sum\limits^{n}_{j=0}x^{2}_{j}$. 


\section{Matice}

Pro sázení matic se velmi často používá prostředí \verb|array| a~závorky (\verb|\left|, \verb|\right|). 

\[\textrm{\bf A}=\left|\begin{array}{cccc}a_{11}&a_{12}&\ldots&a_{1n}\\ a_{21}&a_{22}&\ldots&a_{2n}\\ \vdots&\vdots&\ddots&\vdots\\ a_{m1}&a_{m2}&\ldots&a_{mn}\end{array}\right| = \left|\begin{array}{cc}t&u\\v&w\end{array}\right| =
tw - uv\]

Prostředí \verb|array| lze úspěšně využít i~jinde.

\[\binom{n}{k}=\left\{\begin{array}{cl}
\frac{n!}{k!(n-k)!} &\mbox{pro } 0\leq k\leq n \\[.2em]
0 &\mbox{pro } k>n \mbox{ nebo } k<0 \end{array}\right.\]
\end{document}
